\usepackage{fancyhdr}
\usepackage[%
    a5paper,
%    papersize={5.5in,8.5in},
    %twoside
    margin=0.75in,
    top=0.75in,
    bottom=0.75in,
    ]{geometry}

\usepackage{xcolor}
\usepackage{wrapfig}
\usepackage{graphicx}

% Specialize colors
\definecolor{vegetarianColor}{RGB}{44, 158, 38}
\definecolor{glutenFreeColor}{RGB}{195, 124, 47}
\definecolor{drinkColor}{RGB}{48, 94, 135}
\definecolor{dessertColor}{RGB}{194, 102, 169}
\definecolor{meatColor}{RGB}{164, 10, 10}
\definecolor{fishColor}{RGB}{209, 187, 63}
\definecolor{HBRS}{RGB}{0, 157, 224}

% Change header line
\let\oldheadrule\headrule
\renewcommand{\headrule}{\color{HBRS}\oldheadrule}
\renewcommand{\headrulewidth}{2pt}%

% Headers to the page - dish type, additional dish type, serves how many
\newcommand{\dishType}[1]{\rhead{#1}}
\newcommand{\dishOther}[1]{\lhead{#1}}
\newcommand{\serves}[2][Serves]{\chead{#1 #2}}

% Dish types
\newcommand{\vegetarian}{ {\large\color{vegetarianColor}\textbf{Vegetarian}} }
\newcommand{\vegan}{ {\large\color{vegetarianColor}\textbf{Vegan}} }
\newcommand{\glutenFree}{ {\large\color{glutenFreeColor}\textbf{Gluten Free}} }
\newcommand{\drink}{      {\large\color{drinkColor}\textbf{Drink}} }
\newcommand{\dessert}{    {\large\color{dessertColor}\textbf{Dessert}} }
\newcommand{\maindish}{       {\large\color{meatColor}\textbf{Main Dish}} }
\newcommand{\meat}{       {\large\color{meatColor}\textbf{Meat}} }
\newcommand{\fish}{       {\large\color{fishColor}\textbf{Fish}} }

\newcommand{\makeahead}{  {\large\color{makeaheadcolor}\textbf{M}} }

%% Preparation and cook times:
\newcommand{\prepTime}[2][Prep time]{\lfoot{#1: #2}}
\newcommand{\cookTime}[2][Cook time]{\rfoot{#1: #2}}


\usepackage{xparse}    % for new commands
\makeatletter          % needed for commands below

% From Donald Arseneau. Add after the wrapping text. Whew!
\def\wrapfill{%
    \par
    \ifx\parshape\WF@fudgeparshape
        \nobreak
        \ifnum\c@WF@wrappedlines>\@ne
            \advance\c@WF@wrappedlines\m@ne
            \vskip\c@WF@wrappedlines\baselineskip
            \global\c@WF@wrappedlines\z@
        \fi
        \allowbreak
        \WF@finale
    \fi
}

% Used for the headnote and in \showit
% If the text is small it is placed on one line;
% otherwise it is put into a raggedright paragraph.
\long\def\writeOneLine#1{%
    \sbox\@tempboxa{#1}%
    \ifdim \wd\@tempboxa <0.75\linewidth
        \begingroup
            \itshape
            #1\par
        \endgroup
    \else
       \parbox{0.95\linewidth}{\raggedright\itshape#1}% %% size of comment text
       \par
    \fi
}

\newif\if@mainmatter \@mainmattertrue

%% Borrowed from book.cls
\newcommand\frontmatter{%
    \cleardoublepage
    \@mainmatterfalse
    \pagenumbering{roman}}
    
\newcommand\mainmatter{%
    \cleardoublepage
    \@mainmattertrue
    \pagenumbering{arabic}}

\makeatother

% Thanks to alephzero for the excellent start:
% #1 [optional headnote]; #2 Title of recipe; #3 [Initial instructions]
\newcounter{stepnum}

\NewDocumentCommand{\recipe}{o m o}{%
    \setcounter{stepnum}{0}    % Reset the recipe's steps counter 
    \newpage
    \thispagestyle{fancy}
    \lhead{}%
    \chead{}%
    \rhead{}%
    \lfoot{}%
    \rfoot{}%
    \section{#2}%
    \IfNoValueF{#1}{\begin{center}\writeOneLine{#1}\end{center}}
    \IfNoValueF{#3}{\noindent\emph{#3}\par\medskip}
}

%%\newcommand{\temp}[1]{#1°C}

% include a graphic
\newcommand{\showpic}[3]{%
    \begin{center}
        \bigskip
        \includegraphics[width=#1\textwidth]{#2}
        \par
        \medskip
        \writeOneLine{#3}    % figure caption
        \par
    \end{center}%
}

\def\ucit#1{\uppercase{#1}}
\begingroup
    \lccode`~=`\^^M
    \lowercase{%
\endgroup%% Ingredient first, then measure; empty measure and/or unit = " . "
    %% *=column break; amount<space>ingredient
    \NewDocumentCommand{\ing}{u{ } u{ } u{~}}{% %% basically the same as: \def\ing#1 #2~{% requires xparse
        \noindent
        \if#1#2% Is a heading, a non-ingredient, in the ingredients block
            \emph{#3}~ % A heading
        \else % Amounts containing spaces <1 teaspoon> have to use '~' <1~teaspoon>
            \textbf{\ucit#3, }#1\if.#2\else\ #2\fi~ %
        \fi
    }%
}%

% New command for two columns containing the steps and the corresponding ingredients
% First the ingredients, and then the methodology
\NewDocumentEnvironment{step}{}{%
    \parindent0pt
    \leftskip0pt
    \begin{minipage}{\textwidth}
        \begin{wrapfigure}{r}{0pt}
            \kern-0.5em
            \vrule width 1pt\enskip
            \begin{minipage}{0.5\textwidth}
                \leftskip=1.5em
                \parindent=-1.5em
                \parskip=0.25em
                \obeylines
                \everypar={\ing}
}{%
    \wrapfill
    \end{minipage}
    \medskip
}

\NewDocumentCommand{\method}{}{%
    \end{minipage}
    \end{wrapfigure}
    \rightskip0pt plus 2em
    \parskip0.25em
    \everypar={\llap{\stepcounter{stepnum}\hbox to 1.5em{\thestepnum.\hfill}}}
}


\pagestyle{plain}
\setlength{\intextsep}{0pt}
